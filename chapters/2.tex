\section {Kevin Natanael Nainggolan 1174059}
\subsubsection{Pemahanan Teori}

\begin{enumerate}

    \item Apa itu fungsi, inputan 
fungsi dan kembalian fungsi dengan contoh kode program

    lainnya.

    Fungsi adalah bagian dari program yang dapat digunakan ulang.

    Berikut merupakan contoh fungsi dan cara pemanggilannya

    \lstinputlisting[firstline=9, lastline=15]{src/chapter3/chap3_1174059_teori.py}



    Fungsi dapat membaca parameter, parameter adalah nilai yang disediakan kepada fungsi, dimana nilai ini akan menentukan output yang akan dihasilkan fungsi.

    \lstinputlisting[firstline=18, lastline=41]{src/chapter3/chap3_1174059_teori.py}



    Statemen return digunakan untuk keluar dari fungsi. Kita juga dapat menspesifikasikan nilai kembalian.

    \lstinputlisting[firstline=44, lastline=71]{src/chapter3/chap3_1174059_teori.py}



    \item Apa itu paket dan cara pemanggilan paket atau library dengan contoh kode

    program lainnya.

    Untuk memudahkan dalam pemanggilan fungsi yang di butuhkan, agar dapat dipanggil berulang.

    Cara pemanggilannya

    \lstinputlisting[firstline=74, lastline=76]{src/chapter3/chap3_1174059_teori.py}



    \item Jelaskan Apa itu kelas, apa itu objek, apa itu atribut, apa itu method dan

    contoh kode program lainnya masing-masing.

    kelas merupakan sebuah blueprint yang mepresentasikan objek.

    objek adalah hasil cetakan dadri sebuah kelas.

    method adalah suatu upaya yang digunakan oleh object.

    \lstinputlisting[firstline=79, lastline=89]{src/chapter3/chap3_1174059_teori.py}



    \item Jelaskan cara pemanggikan library kelas dari instansiasi dan pemakaiannya den-

    gan contoh program lainnya.

    Cara Pemanggilanya 

    \lstinputlisting[firstline=74, lastline=76]{src/chapter3/chap3_1174059_teori.py}



    \item Jelaskan dengan contoh pemakaian paket dengan perintah from kalkulator im-

    port Penambahan disertai dengan contoh kode lainnya.

    Penggunaan paket from namafile import, itu berfungsi untuk memanggil file dan fungsinya

    \lstinputlisting[firstline=79, lastline=89]{src/chapter3/chap3_1174059_teori.py}



    \item Jelaskan dengan contoh kodenya, pemakaian paket fungsi apabila 
le library

    ada di dalam folder.

    Pemakaian paket adalah perkumpulan fungsi-fungsi. contoh kodenya adalah sebagai berikut :
\lstinputlisting[firstline=92, lastline=92]{src/chapter3/chap3_1174059_teori.py}


    \item Jelaskan dengan contoh kodenya, pemakaian paket kelas apabila 
le library ada

    di dalam folder.

    \lstinputlisting[firstline=95, lastline=95]{src/chapter3/chap3_1174059_teori.py}



\end{enumerate}

\subsubsection{Ketrampilan Pemrograman}

\begin{enumerate}

    \item Buatlah fungsi dengan inputan variabel NPM, dan melakukan print luaran huruf

    yang dirangkai dari tanda bintang, pagar atau plus dari NPM kita. Tanda

    bintang untuk NPM mod 3=0, tanda pagar untuk NPM mod 3 =1, tanda plus

    untuk NPM mod3=2.

\lstinputlisting [firstline=19, lastline=64]{src/chapter3/chap3_1174059_3lib.py}



    \item Buatlah fungsi dengan inputan variabel berupa NPM. kemudian dengan meng-

    gunakan perulangan mengeluarkan print output sebanyak dua dijit belakang

    NPM.

\lstinputlisting [firstline=68, lastline=78]{src/chapter3/chap3_1174059_3lib.py}



    \item Buatlah fungsi dengan dengan input variabel string bernama NPM dan beri

    luaran output dengan perulangan berupa tiga karakter belakang dari NPM se-

    banyak penjumlahan tiga dijit tersebut.

\lstinputlisting [firstline=82, lastline=89]{src/chapter3/chap3_1174059_3lib.py}



    \item Buatlah fungsi hello word dengan input variabel string bernama NPM dan

    beri luaran output berupa digit ketiga dari belakang dari variabel NPM meng-

    gunakan akses langsung manipulasi string pada baris ketiga dari variabel NPM.

\lstinputlisting [firstline=93, lastline=96]{src/chapter3/chap3_1174059_3lib.py}



    \item buat fungsi program dengan input variabel NPM dan melakukan print nomor npm satu persatu kebawah.

\lstinputlisting [firstline=100, lastline=102]{src/chapter3/chap3_1174059_3lib.py}



    \item Buatlah fungsi dengan inputan variabel NPM, didalamnya melakukan penjum-

    lahan dari seluruh dijit NPM tersebut, wajib menggunakan perulangan dan

    atau kondisi.

\lstinputlisting [firstline=106, lastline=110]{src/chapter3/chap3_1174059_3lib.py}



    \item Buatlah fungsi dengan inputan variabel NPM, didalamnya melakukan melakukan

    perkalian dari seluruh dijit NPM tersebut, wajib menggunakan perulangan dan

    atau kondisi.

\lstinputlisting [firstline=114, lastline=118]{src/chapter3/chap3_1174059_3lib.py}



    \item Buatlah fungsi dengan inputan variabel NPM, Lakukan print NPM anda tapi

    hanya dijit genap saja. wajib menggunakan perulangan dan atau kondisi.

\lstinputlisting [firstline=122, lastline=126]{src/chapter3/chap3_1174059_3lib.py}



    \item Buatlah fungsi dengan inputan variabel NPM, Lakukan print NPM anda tapi

    hanya dijit ganjil saja. wajib menggunakan perulangan dan atau kondisi.

\lstinputlisting [firstline=130, lastline=134]{src/chapter3/chap3_1174059_3lib.py}



    \item Buatlah fungsi dengan inputan variabel NPM, Lakukan print NPM anda tapi

    hanya dijit yang termasuk bilangan prima saja. wajib menggunakan perulangan

    dan atau kondisi.

   \lstinputlisting [firstline=138, lastline=150]{src/chapter3/chap3_1174059_3lib.py}



    \item Buatlah satu library yang berisi fungsi-fungsi dari nomor diatas dengan nama

    
le 3lib.py dan berikan contoh cara pemanggilannya pada 
le main.py.

    \lstinputlisting[firstline=8, lastline=8]{src/chapter3/chap3_1174059_main.py}



    \item Buatlah satu library class dengan nama 
le kelas3lib.py yang merupakan mod-

    i
kasi dari fungsi-fungsi nomor diatas dan berikan contoh cara pemanggilannya

    pada 
le main.py.

    \lstinputlisting[firstline=8, lastline=14]{src/chapter3/chap3_1174059_main.py}

    

\end{enumerate}

\subsubsection{Ketrampilan Penanganan Error}

Error yang di dapat dari mengerjakan tugas ini adalah type error, cara menaggulaginya dengan cara mengecheck kembali codingannya

kemudian run kembali aplikasinya

berikut contoh Penggunaan fungsi try dan exception

\lstinputlisting[firstline=8, lastline=14]{src/chapter3/chap3_1174059_teori.py}

\section{Alit Fajar Kurniawan 1174057}
\subsection{Teori}
\begin{enumerate}
    \item Jenis jenis Variable phyton dan cara pemakaiannya 
	Variabel merupakan tempat menyimpan data, Dalam Phyton terdapat beberapa variabel dengan berbagai type data diantaranya adalah variabel dengan type data number, string, dan boolean. 
	Dalam phyton kita dapat membuat variable dengan cara sebagai gambar berikut
    \lstinputlisting[firstline=8, lastline=12]{src/chapter2/1174057_teori.py}
    \item Kode untuk meminta input dari user dan bagaimana melakukan output ke layar
	\lstinputlisting[firstline=67, lastline=68]{src/chapter2/1174057_teori.py}
    \item Operator dasar aritmatika
	Ada operator penambahan, pengurangan perkalian, perkalian, pembagian, modulus, perpangkatan, dan pembulatan decimal.
	\lstinputlisting[firstline=71, lastline=94]{src/chapter2/1174057_teori.py}
    \item Perulangan
	Terdapat dua jenis perulangan di dalam phyton yaitu perulangan while dan perulangan for
	\lstinputlisting[firstline=97, lastline=99]{src/chapter2/1174057_teori.py}
	\lstinputlisting[firstline=102, lastline=105]{src/chapter2/1174057_teori.py}
    \item sintak Untuk memilih kondisi, dan kondisi didalam kondisi
	Pengambilan kondisi If yang digunakan untuk mengantisipasi kondisi yang terjadi saat program dijalankan dan menentukan tindakan apa yang akan diambil sesuai dengan kondisi.
	\lstinputlisting[firstline=108, lastline=111]{src/chapter2/1174057_teori.py}
	\lstinputlisting[firstline=114, lastline=119]{src/chapter2/1174057_teori.py}
	\lstinputlisting[firstline=122, lastline=129]{src/chapter2/1174057_teori.py}

    \item Jenis-jenis error pada phyton
Syntax Errors adalah keadaan dimana kode python mengalami kesalahan penulisan. 
ZeroDivisonError adalah eror yang terjadi saat eksekusi program menghasilkan perhitungan matematika pembagian dengan angka nol.
NameError adalah eror yang terjadi saat kode di eksekusi terhadap local name atau global name yang tidak terdefinisi. 
TypeError adalah eror yang terjadi saat dilakukan eksekusi pada suatu operasi atau fungsi dengan type object yang tidak sesuai.

    \item Cara memakai try except
Cara pemakaian try except adalah sebagai berikut :
	\lstinputlisting[firstline=132, lastline=138]{src/chapter2/1174057_teori.py}

\end{enumerate}

\subsection{praktek}
\begin{enumerate}
    \item Jawaban soal no 1
    \lstinputlisting[firstline=11, lastline=20]{src/chapter2/1174057_praktek.py}
    \item Jawaban soal no 2
    \lstinputlisting[firstline=24, lastline=28]{src/chapter2/1174057_praktek.py}
    \item Jawaban soal no 3
    \lstinputlisting[firstline=33, lastline=37]{src/chapter2/1174057_praktek.py}
    \item Jawaban soal no 4
    \lstinputlisting[firstline=40, lastline=41]{src/chapter2/1174057_praktek.py}
    \item Jawaban soal no 5
    \lstinputlisting[firstline=44, lastline=56]{src/chapter2/1174057_praktek.py}
    \item Jawaban soal no 6
    \lstinputlisting[firstline=59, lastline=60]{src/chapter2/1174057_praktek.py}
    \item Jawaban soal no 7
    \lstinputlisting[firstline=63, lastline=64]{src/chapter2/1174057_praktek.py}
    \item Jawaban soal no 8
    \lstinputlisting[firstline=67, lastline=71]{src/chapter2/1174057_praktek.py}
    \item Jawaban soal no 9
    \lstinputlisting[firstline=74, lastline=74]{src/chapter2/1174057_praktek.py}
    \item Jawaban soal no 10
    \lstinputlisting[firstline=77, lastline=77]{src/chapter2/1174057_praktek.py}
    \item Jawaban soal no 11
    \lstinputlisting[firstline=80, lastline=80]{src/chapter2/1174057_praktek.py}
\end{enumerate}

\subsection{Keterampilan dan penanganan eror}
    \lstinputlisting[firstline=10, lastline=17]{src/errr2.py}

\section{Muhammad Iqbal Panggabean}
\subsection{Teori}
\begin{enumerate}
    \item Jenis jenis variable phyton dan cara pemakaiannya
Variabel merupakan tempat menyimpan data. Dalam Phyton terdapat beberapa variabel dengan berbagai type data diantaranya adalah variabel dengan type data number, string, dan boolean. Dalam phyton kita dapat membuat variable dengan cara sebagai gambar berikut
   \lstinputlisting[firstline=8, lastline=12]{src/1174063_teori.py}
    \item Kode untuk meminta input dari user dan bagaimana melakukan output ke layar
 \lstinputlisting[firstline=67, lastline=68]{src/1174063_teori.py}
    \item Operator dasar aritmatika
Ada operator penambahan, pengurangan perkalian, perkalian, pembagian, modulus, perpangkatan, dan pembulatan decimal.
\lstinputlisting[firstline=71, lastline=94]{src/1174063_teori.py}
    \item Perulangan
Terdapat dua jenis perulangan di dalam phyton yaitu perulangan while dan perulangan for
 \lstinputlisting[firstline=97, lastline=99]{src/1174063_teori.py}
 \lstinputlisting[firstline=102, lastline=105]{src/1174063_teori.py}
    \item sintak Untuk memilih kondisi, dan kondisi didalam kondisi
Pengambilan kondisi If yang digunakan untuk mengantisipasi kondisi yang terjadi saat program dijalankan dan menentukan tindakan apa yang akan diambil sesuai dengan kondisi.
  \lstinputlisting[firstline=108, lastline=111]{src/1174063_teori.py}
  \lstinputlisting[firstline=114, lastline=119]{src/1174063_teori.py}
  \lstinputlisting[firstline=122, lastline=129]{src/1174063_teori.py}

    \item Jenis-jenis error pada phyton
Syntax Errors adalah keadaan dimana kode python mengalami kesalahan penulisan. 
ZeroDivisonError adalah eror yang terjadi saat eksekusi program menghasilkan perhitungan matematika pembagian dengan angka nol.
NameError adalah eror yang terjadi saat kode di eksekusi terhadap local name atau global name yang tidak terdefinisi. 
TypeError adalah eror yang terjadi saat dilakukan eksekusi pada suatu operasi atau fungsi dengan type object yang tidak sesuai.

    \item Cara memakai try except
Cara pemakaian try except adalah sebagai berikut :
\lstinputlisting[firstline=132, lastline=138]{src/1174063_teori.py}

\end{enumerate}

\subsection{praktek}
\begin{enumerate}
    \item Jawaban soal no 1
    \lstinputlisting[firstline=11, lastline=20]{src/1174063_praktek.py}
    \item Jawaban soal no 2
    \lstinputlisting[firstline=24, lastline=28]{src/1174063_praktek.py}
    \item Jawaban soal no 3
    \lstinputlisting[firstline=33, lastline=37]{src/1174063_praktek.py}
    \item Jawaban soal no 4
    \lstinputlisting[firstline=40, lastline=41]{src/1174063_praktek.py}
    \item Jawaban soal no 5
    \lstinputlisting[firstline=44, lastline=56]{src/1174063_praktek.py}
    \item Jawaban soal no 6
    \lstinputlisting[firstline=59, lastline=60]{src/1174063_praktek.py}
    \item Jawaban soal no 7
    \lstinputlisting[firstline=63, lastline=64]{src/1174063_praktek.py}
    \item Jawaban soal no 8
    \lstinputlisting[firstline=67, lastline=71]{src/1174063_praktek.py}
    \item Jawaban soal no 9
    \lstinputlisting[firstline=74, lastline=74]{src/1174063_praktek.py}
    \item Jawaban soal no 10
    \lstinputlisting[firstline=77, lastline=77]{src/1174063_praktek.py}
    \item Jawaban soal no 11
    \lstinputlisting[firstline=80, lastline=80]{src/1174063_praktek.py}
\end{enumerate}

\subsection{Keterampilan dan penanganan eror}
    \lstinputlisting[firstline=10, lastline=17]{src/errr2.py}