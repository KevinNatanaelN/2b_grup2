\section{Yusniar Nur Syarif Sidiq/1164089}
\subsection{Teori}
\begin{enumerate}
\item Fungsi merupakan sebuah bagian dari program yang dapat digunakan ulang dan memiliki inputan variabel serta nilai yang akan di kembalikannya. Contohnya adalah source code berikut ini,
	 \lstinputlisting{src/chapter2/1164089/1164089_1.py}
Dalam dalam source code tersebut akan mengeluarkan output Hallo 1164089 ketika kita running di dalam spyder.

\item Library dalam python disini merupakan kumpulan dari fungsi dan cara pemanggilannya adalah dengan melakukan import file librarynya. Sebagai contoh, buatlah Matematika.py dan 1164089\_2.py, simpan dalam satu folder. Untuk Matematika.py isikan fungsi sebagai berikut
	 \lstinputlisting{src/chapter2/1164089/Matematika.py}
Untuk memanggil fungsi tersebut adalah dengan melakukan import  Matematika.py pada 1164089\_2.py adalah sebagai berikut,
	 \lstinputlisting{src/chapter2/1164089/1164089_2.py}

\item Class merupakan salah satu cara untuk membuat sebuah kode yang mempunyai objek serta atribut tertentu sehingga akan lebih mudah dalam mengorganisasi berbagai fungsi dan statenya. Objek disini merupakan instansiasi atau perwujudan dari sebuah class. Untuk membuat class yang memiliki objek serta atribut dapat dilihat pada source code berikut ini, dimana kita akan membaut file bernama mtk.py
	\lstinputlisting{src/chapter2/1164089/mtk.py}
Self tersebut berfungsi untuk menunjukkan variabel lokal dari class tersebut. Untuk memanggil class tersebut kita akan membuat file bernama 1164089\_3.py dan kita akan melakukan import mtk.py pada file tersebut, untuk source codenya dapat dilihat seperti berikut,
	\lstinputlisting{src/chapter2/1164089/1164089_3.py}

\item Cara memanggilnya yaitu
	\begin{itemize}
		\item Pertama import terlebih dahulu filenya
		\item Buat variabel yang berfungsi menampung data
		\item Panggil nama classnya dan methodnya
		\item Gunakan perintah print untuk menampilkannya
	\end{itemize}
Sebagai contoh perhatikan source code ini,
	\lstinputlisting{src/chapter2/1164089/1164089_4.py}

\item Dimana kita akan melakukan membuka library Matematika.py dan akan melakukan import dari fungsi di dalamnya yaitu matematika, sehingga akan lebih simple dalam penulisan source codenya adalah sebagai berikut,
	\lstinputlisting{src/chapter2/1164089/1164089_5.py}

\item

\item 
\end{enumerate}

\subsection{Keterampilan Pemrograman}
\begin{enumerate}

\item Soal No 1 \lstinputlisting{src/chapter2/1164089/1164089_21.py}

\item Soal No 2 \lstinputlisting{src/chapter2/1164089/1164089_22.py}

\item Soal No 3 \lstinputlisting{src/chapter2/1164089/1164089_23.py}

\item Soal No 4 \lstinputlisting{src/chapter2/1164089/1164089_24.py}

\item Soal No 5 \lstinputlisting{src/chapter2/1164089/1164089_25.py}

\item Soal No 6 \lstinputlisting{src/chapter2/1164089/1164089_26.py}

\item Soal No 7 \lstinputlisting{src/chapter2/1164089/1164089_27.py}

\item Soal No 8 \lstinputlisting{src/chapter2/1164089/1164089_28.py}

\item Soal No 9 \lstinputlisting{src/chapter2/1164089/1164089_29.py}

\item Soal No 10 \lstinputlisting{src/chapter2/1164089/1164089_30.py}

\item Soal No 11 \lstinputlisting{src/chapter2/1164089/1164089_31.py}

\item Soal No 12 \lstinputlisting{src/chapter2/1164089/1164089_32.py}
\end{enumerate}

\subsection{Penanganan Erorr}
\begin{enumerate}

\item Erorr yang saya temui di antaranya adalah Systax Erorr, dimana suatu keadaan script python mengalami kesalahan dalam penulisannya dan solusi dari permasalahan ini adalah dengan memperbaiki script penulisan yang salah. Untuk contoh fungsi trx except dapat dilihat pada source code berikut ini,

	\lstinputlisting{src/chapter2/1164089/1164089_33.py}

\end{enumerate}

\section{Fathi Rabbani/1164074/3C}
\subsection{Teori}
\begin{enumerate}
\item Fungsi, Inputan dan Return
	\begin{itemize}
	\item Fungsi adalah sebuah blok code yang digunakan untuk melempar parameter kedalam blok code yang berbeda.
		\lstinputlisting[firstline=9, lastline=11]{src/chapter3/1164074/praktek.py}
	\item Inputan fungsi adalah sebuah fungsi yang memiliki parameter berupa inputan atau data yang bias diinputkan.
		\lstinputlisting[firstline=13, lastline=15]{src/chapter3/1164074/praktek.py}
	\item Pengembalian fungsi atau sering juga disebut sebagai return merupakan sebuah pengembalian nilai dari pengeksekusian data pada parameter yang terdapat difungsi.
		\lstinputlisting[firstline=17, lastline=24]{src/chapter3/1164074/praktek.py}
	\end{itemize}

\item Paket atau Library Fungsi
	\subitem paket merupakan sebuah penggunaan library dengan maksud mempermudah dalam eksekusi dan pemanggilan fungsi
		\lstinputlisting[firstline=171, lastline=173]{src/chapter3/1164074/praktek.py}
	berikut ini adalah fungsi yang digunakan untuk memanggil paket atau library
		\lstinputlisting[firstline=175, lastline=178]{src/chapter3/1164074/praktek.py}
\item Kelas, Objek, Atribut dan Method
	\begin{itemize}
	\item kelas merupakan sebuah blueprint dari objek
	\item objek merupakan sebuah data hasil eksekusi dari kelas
	\item atribut merupakan nilai data yang terdapat didalam objek
	\item method merupakan operasi atau eksekusi yang dilakukan dengan data dari objek
		\lstinputlisting[firstline=1, lastline=7]{src/chapter3/1164074/praktek.py}
	\end{itemize}
\item Library Kelas
	\begin{itemize}
	\item Penggunaan kelas dan datanya
		\lstinputlisting[firstline=1, lastline=7]{src/chapter3/1164074/praktek.py}
	\item Pemanggilan library kelas
		\lstinputlisting[firstline=181, lastline=186]{src/chapter3/1164074/praktek.py}
	\end{itemize}
\item Pemanggilan Library Kalkulator
	\begin{itemize}
	\item data Kalkulator
		\lstinputlisting[firstline=189, lastline=191]{src/chapter3/1164074/praktek.py}
	\item data pemanggilan
		\lstinputlisting[firstline=193, lastline=198]{src/chapter3/1164074/praktek.py}
	\end{itemize}
\item Penggunaan Paket Fungsi
	\begin{itemize}
	\item data Fungsi Kalkulator
		\lstinputlisting[firstline=189, lastline=191]{src/chapter3/1164074/praktek.py}
	\item data Pemanggil Fungsi
		\lstinputlisting[firstline=200, lastline=203]{src/chapter3/1164074/praktek.py}
	\end{itemize}
\item Penggunaan Paket Kelas
	\begin{itemize}
	\item data Kelas fthr dari file praktek
		\lstinputlisting[firstline=1, lastline=7]{src/chapter3/1164074/praktek.py}
	\item data pemanggil
		\lstinputlisting[firstline=183, lastline=188]{src/chapter3/1164074/praktek.py}
	\end{itemize}
\end{enumerate}

\subsection{Praktek Pemrograman}
	\begin{enumerate}
	\item\lstinputlisting[firstline=27, lastline=74]{src/chapter3/1164074/praktek.py}
	\item\lstinputlisting[firstline=77, lastline=82]{src/chapter3/1164074/praktek.py}
	\item\lstinputlisting[firstline=85, lastline=96]{src/chapter3/1164074/praktek.py}
	\item\lstinputlisting[firstline=99, lastline=103]{src/chapter3/1164074/praktek.py}
	\item\lstinputlisting[firstline=106, lastline=110]{src/chapter3/1164074/praktek.py}
	\item\lstinputlisting[firstline=113, lastline=119]{src/chapter3/1164074/praktek.py}
	\item\lstinputlisting[firstline=122, lastline=128]{src/chapter3/1164074/praktek.py}
	\item\lstinputlisting[firstline=131, lastline=138]{src/chapter3/1164074/praktek.py}
	\item\lstinputlisting[firstline=141, lastline=147]{src/chapter3/1164074/praktek.py}
	\item\lstinputlisting[firstline=150, lastline=158]{src/chapter3/1164074/praktek.py}
	\item
		\begin{itemize}
		\item data 3lib.py
			\lstinputlisting[firstline=17, lastline=191]{src/chapter3/1164074/3lib.py}
		\item data main.py
			\lstinputlisting[firstline=1, lastline=5]{src/chapter3/1164074/main.py}
		\end{itemize}
	\end{enumerate}

\subsection{Handling Error}
\par Error yang di dapat dari mengerjakan tugas ini adalah type error, cara menaggulaginya dengan cara mengecheck kembali codingannya, kemudian run kembali aplikasinya berikut adalah contoh Penggunaan fungsi try dan exception :
\lstinputlisting[firstline=161, lastline=168]{src/chapter3/1164074/praktek.py}